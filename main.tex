\documentclass[onecolumn]{aastex63}
\usepackage{natbib}
%\definecolor{orcidlogocol}{HTML}{A6CE39}
\bibliographystyle{aasjournal}

\begin{document}

\title{COLLISION RATES OF PLANETESIMALS NEAR MEAN-MOTION RESONANCES}

\author{Spencer C. Wallace}
\affiliation{Astronomy Department, University of Washington, Seattle, WA 98195}

\author{Aaron C. Boley}
\affiliation{Department of Physics and Astronomy, University of British Columbia, Vancouver BC, Canada}

\author{Thomas R. Quinn}
\affiliation{Astronomy Department, University of Washington, Seattle, WA 98195}

\begin{abstract}
In circumstellar disks, collisional grinding of planetesimals produces second-generation dust, which can be observed through thermal 
emission. While it remains unclear when second-generation dust first becomes a major component of the total dust content, the presence of 
such dust and potentially the substructure within it can be used to explore a disk's physical conditions. A perturbing planet has been shown to 
produce nonaxisymmetric structures, as well as gaps in disks, regardless of the origin of the dust. However, small grains will have very 
different dynamics compared with planetesimals when in the presence of gas, and as such, the collisional evolution of planetesimals could 
create dusty disk structures that would not exist otherwise. In particular, mean motion resonances (MMRs) are extremely nonlinear and could 
drive disk morphologies. We use a direct N-body model to track collision rates in a planetesimal disk under the gravitational influence of an 
external giant planet. We show that a perturbing planet can produce significant variations in the collision rate of planetesimals near the 
MMRs and we explore how the mass and eccentricity of the perturbing planet alters this structure. Finally, we use the CASA image simulator to 
predict what the dust emission sturcture should look like with ALMA and the NG-VLA. Although the dust structure produced by MMRs cannot 
be used to directly measure the properties of the perturbing planet, we show that it can be used to place constraints on its mass and 
eccentricity.
\end{abstract}

\section{Introduction} \label{sec:intro}

Recent observations of circumstellar disks by ALMA have revealed a rich variety of substructure in the millimeter wavelength
contiunuum emission. Features such as gaps and asymmetries 
\citep{2015ApJ...808L...3A, 2016Sci...353.1519P, PhysRevLett.117.251101, 2016ApJ...820L..40A, 2016Natur.535..258C} in the 
emission provide diagnostics for the physical processes that drive the evolution of the disks. At these wavelengths, much of this light 
is produced by second-generation debris generated through collisional grinding of planetesimals
\citep[see][]{2008ARA&A..46..339W}. 

In some cases, these gap features are argued to be an indicator for the presence of a giant planet, either embedded in the disk 
\citep{2015MNRAS.453L..73D} or orbiting externally. An giant perturber can influence the structure of the dust emission in a number 
of ways. A misaligned giant planet can produce nonaxisymmetric features such as warps \citep{2001A&A...370..447A}. Highly 
eccentric perturbers can produce even more complicated structures through secular perturbations \citep{2014MNRAS.443.2541P, 
2015MNRAS.448.3679P}. Mean motion resonances (MMRs) have been shown to open gaps as well
\citep{2015ApJ...798...83N, 2016ApJ...818..159T, 2018ApJ...857....3T}.

Also mention \citet{2020A&A...635A..10S}, looked at dust trapping in resonances with terrestrial SS planets. Also mention 
\citet{2020arXiv200408736G}, detection of a planetesimal collision around Fomalhut.

The dynamics governing the motion of bodies near MMRs is extremely nonlinear, as is determining what the collision rates between 
planetesimals should look like in these regions. For a collection of bodies massive enough to experience the effects of gravitational 
focusing, a large eccentricity dispersion tends to reduce the probability of collision, while enhancements in surface density tends to 
increase it. Due to conservation of the Jacobi energy, MMRs simultaneously enhance the local eccentricity dispersion and also 
enhance the surface density adjacent to the resonance \citep{2000Icar..143...45R, 2017ApJ...850..103B}.

Because the second generation dust production is driven by the planetesimal collision rate, it is crucial to understand how the 
dynamics that drive collisions works. In particular, it is not obvious how to link the readily observable thermal emission from dust in 
protoplanetary disks to the presence of a perturbing giant planet. Due to its nonlinearity, this problem is best studied with N-body 
simulations. Unfortunately, collision detection in an N-body simulation is extremely computationally expensive. So far, studies of 
planetesimal dynamics near MMRs have involved either collisionless test particles \citep{2017ApJ...850..103B, 2016ApJ...818..159T, 
2018ApJ...857....3T} or severely limited integration times \citep{2000Icar..143...45R}.

To further elucidate this subject, we use the tree based N-body code {\sc ChaNGa}
\citep{2008IEEEpds...ChaNGa, 2015AphCom..2..1} to follow the collisional evolution of a planetesimal disk under the gravitational 
influence of a Jupiter sized body. Because particle positions are sorted into a tree structure, neighbor finding and collision detection 
can be done quickly and efficiently. This considerably relaxes the constraints on resolution and integration time. With this toolset, we 
explore the collision rate structure of the planetesimal disk in the vicinity of mean motion resonances. In particular, we would like to 
determine: 1. whether MMRs leave a detectable signature in the collisionally-generated dust. 2. If these signatures can be used to determine or constrain the orbital properties of the perturbing planet.

This work is organized in the following way: In section \ref{sec:dynamics}, we provide an overview of the relevant dynamics that drive 
the evolution of a planetesimal disk under the gravitational influence of an external perturber. We also highlight the shortcomings of secular theory in trying to predict the radial collision rate and motivate the need for N-body simulations with massive, collisional particles. In section \ref{sec:sims} we describe the N-body code that we use and detail the initial conditions that are used for two simulations: one in which the perturber is on a circular orbit and another in which the perturber is given a mildly eccentric orbit. Section \ref{sec:results} presents the results of the two simulations. Next, we generate dust emission maps in section \ref{sec:dust}, under the assumption that collisionally-generated dust will quickly couple with the gas and circularize, and show that radial structure is produced in the vicinity of the MMRs. Using the dust emission maps, section \ref{sec:fitting} explores the feasibility of measuring properties of the perturbing planet from the MMR features in the dust. We then conclude in section \ref{sec:discuss}. 

\section{Overview of Relavant Dynamics} \label{sec:dynamics}

\subsection{Secular Forcing}\label{sec:sec_force}

The most direct and widespread effect that a giant perturber will have on a planetesimal disk is through secular forcing of the 
eccentricities of the planetesimals. This will cause the complex eccentricities of the planetesimals to take on a time independent 
forced value, given by \citep{1999ApJ...527..918W} as

\begin{equation}\label{eq:eforced}
	z_{f} = \frac{b^{2}_{3/2} (\alpha)}{b^{1}_{3/2} (\alpha)} e_{g} ~ \mathrm{exp} ~ i \omega_{g}.
\end{equation}

\noindent Here, $\alpha = a_{g} / a$ where $a_{g}$ and $a$ are the semi-major axes of the giant planet and the planetesimal, 
respectively. $e_{g}$ and $\omega_{g}$ are the eccentricity and longitude of pericenter of the giant and $b^{j}_{s} (\alpha)$ is a 
Laplace coefficient given by \citep{2000ssd..book.....M} as

\begin{equation}\label{eq:lap}
	b_{s}^{j}(\alpha) = \frac{1}{2 \pi} \int_{0}^{2 \pi} \frac{cos \, j \theta \, d \theta}{\left( 1 - 2 \alpha \, cos \theta + \alpha^2 \right)^{s}}.
\end{equation}

Without any nearby secular or mean motion resonances, equation \ref{eq:eforced} will completely describe the eccentricities and longitude of  
pericenter orientations of the planetesimals. Additional forces due to two-body scattering between planetesimals, along with 
aerodynamic gas drag will add an additional free component to the complex eccentricity, which will be randomly oriented. The 
magnitude of the free eccentricity describes how dynamically hot the planetesimal disk is and sets the random encounter speeds of 
planetesimals. When the dynamical excitation of the disk is driven by gravitational stirring, the magnitude of the free eccentricity can 
be described by a Rayleigh distribution \citep{1992Icar...96..107I}.

\subsection{Mean Motion Resonances}

In regions where there are commensurabilities between frequencies, secular theory breaks down and bodies are subject to strong 
perturbations. For the purposes of this study, we will ignore secular resonances, which generally occur on rather large timescales 
and will focus on mean motion resonances (MMRs). A MMR occurs  when the orbital period ratio between two bodies is sufficiently 
close to

\begin{equation}\label{eq:per_mmr}
	\frac{P}{P'} = \frac{p + q}{p},
\end{equation}

\noindent where  $p$ and $q$ are integers $>$ 0 and the unprimed and primed quantities correspond to the perturber and the body 
being perturbed, respectively. If the perturber is much more massive than the other body, the condition for MMR is set  by the semi-major 
axes of the two bodies by

\begin{equation}\label{eq:a_mmr}
	\frac{a}{a'} = \left( \frac{p}{p + q} \right)^{2/3}.
\end{equation}

If we further assume that the two bodies are orbiting in the same plane, the behavior of the bodies near resonance is determined by the
behavior of a critical angle

\begin{equation}\label{eq:phi_crit}
	\phi = (p + q) \lambda' - q \lambda - (2 p + q) \varpi,
\end{equation}

\noindent where $\lambda = \varpi + M$ is the mean longitude of a body. For bodies in resonance, the critical angle will librate around an 
equlibrium value, while this angle will circulate outside of resonance. This behavior is analogous to the motion of a pendulum. Changes in the 
critical angle are coupled to changes in the mean motion and semimajor axis \citet{2000ssd..book.....M}. An important point to note, which will 
become important later, is that the circulation frequency approaches zero near the edge of a resonance. The width of a resonance can be 
defined by determining the largest variation in semimajor axis that permits librational, rather than circulatory motion of the pendulum. For 
second and higher order resonances, the maximum libration width is given by

\begin{equation}\label{eq:res_so}
	\frac{\delta a}{a} = \pm \left( \frac{16}{3} \frac{\left| C_{r} \right|}{n} e^{\left| 2 p + q \right|} \right)^{1/2},
\end{equation}

\noindent where n, a and e are the mean motion, semimajor axis and eccentricity of the unperturbed body. $C_{r}$ is a constant defined by $m'/m_{c} n \alpha f_{d}(\alpha)$, where $\alpha$ = $a/a'$ and $f_{d}$ is the resonant part of the disturbing function. For an interior second-order resonance,

\begin{equation}\label{eq:fd_so}
	f_{d} (\alpha) = \frac{1}{8} \left[ -5(p+q) + 4(p+q)^{2} - 2 \alpha D + 4(p+q) \alpha D + \alpha^{2} D^{2} \right] b^{p+q}_{1/2},
\end{equation}

\noindent where $D b^{j}_{s}$ is the first derivative with respect to $\alpha$ of the Laplace coefficient defined in equation \ref{eq:lap}. This can 
be written as

\begin{equation}\label{eq:lap_d}
	\frac{d b_{s}^{j}}{d \alpha} = s \left( b_{s+1}^{j-1} - 2 \alpha b_{s+1}^{j} + b_{s+1}^{j+1} \right).
\end{equation}

For a first order resonance, the associated disturbing function terms are slightly simpler

\begin{equation}\label{eq:fd_fo}
	f_{d}(\alpha) = (p+q) b_{1/2}^{p+q} + \frac{\alpha}{2} D b_{1/2}^{p+q},
\end{equation}

\noindent and there is an additional contribution to the motion of the critical angle by the pericenter precession rate such that

\begin{equation}\label{eq:res_fo}
	\frac{\delta a}{a} = \pm \left(\frac{16}{3} \frac{\left| C_{r} \right|}{n} e \right)^{1/2} \left(  1 + \frac{1}{27 p^2 e^3} \frac{\left| C_{r} \right|}{n} 
	\right)^{1/2} - \frac{2}{9 p e}  \frac{\left| C_{r} \right|}{n}.
\end{equation}

Changes in eccentricity are correlated with changes in semimajor axis variations according to the Tisserand relation

\begin{equation}\label{eq:tiss}
	\frac{de}{da} = \frac{a^{3/2} - 1}{2 a^{5/2} e}.
\end{equation}

The net result of this is that MMRs will produce 'spikes' in the semimajor axis - eccentricity distribution of a planetesimal disk. In order to 
preserve the Jacobi energy, equation \ref{eq:tiss} predicts that these spikes will tend to curve away from the perturbing body. This lowers the 
surface density of the disk near the locations of resonances, while producing a pileup of bodies on eccentric orbits adjacent to the MMR. The 
enhanced surface density will increase the collision rate. At the same time, the large eccentricites will either lower it due to the decreased 
effectiveness of gravitational focusing, or raise it due to the more vigorous encounter rate, depending on how large the encounter velocity gets 
compared to the escape velocity of the planetesimals. In addition, the MMRs cause the orbits of planetesimals to precess on relatively short 
timescales, which effectively randomizes their longitudes of perihelion. In a secularly forced disk, there will be an interface between aligned 
and randomly oriented orbits at the edge of the resonance. Due to the complicated dynamics that are at play in this region, we turn to a direct 
N-body treatment to better understand how the collision rate varies in the vicinity of a MMR.

\section{Simulations} \label{sec:sims}

\subsection{Numerical Methods}\label{sec:methods}

To follow the dynamical and collisional evolution of a planetesimal disk, we use the highly parallel N-body code {\sc ChaNGa} 
\footnote{A public version of {\sc ChaNGa} can be downloaded from \url{http://www-hpcc.astro.washington.edu/tools/ChaNGa.html}}. 
This code, which is written in the {\sc CHARM++} parallel programming language, was originally designed for cosmology simulations 
and has been shown to perform well on up to half a million processors \citep{2015AphCom..2..1}. {\sc ChaNGa} calculates 
gravitational forces using a modified Barnes-Hut \citep{1986Natur.324..446B} tree with hexadecapole expansions of the moments. 
All of the simulations we perform use a node opening criterion of $\theta_{BH}$ = 0.7. More information about the code can be found 
in \citet{2008IEEEpds...ChaNGa}. We have also recently modified {\sc ChaNGa} to handle solid-body collisions. A full description of 
the collision module implementation can be found in \citet{2019MNRAS.489.2159W}.

\subsection{Initial Conditions}\label{sec:ics}

In total, five simulations are run. The first is a 'nominal' case, in which the perturbing planet's mass and eccentricity are set to that of Jupiter's. 
In the other four cases, the mass (eccentricity) is increased or decreased by a factor of two, while the mass (eccentricity) is held at the nominal 
value. In all cases, the perturbing giant is placed on a 5.2 AU orbit around a 1 $M_{\odot}$ star. The planetesimal disk extends from 2.2 to 3.8 
AU, which covers the two most prominent mean motion resonances with the giant planet, the 2:1 at 3.27 AU and the 3:1 at 2.5 AU.The 
planetesimal disk follows a minimum-mass solar nebula surface density profile \citep{1981PThPS..70...35H}

\begin{equation}\label{eq:surf_den}
	\Sigma = \Sigma_{0} r^{-\alpha},
\end{equation}

\noindent with $\alpha$ = 3/2 and $\Sigma_{0}$ = 10 g cm$^{-2}$. Planetesimals are given a bulk density of 2 g cm$^{-3}$ and a diameter of 
300 km. This corresponds to a disk containing roughly 500,000 bodies. Each simulation is evolved for 5,000 years, which is about 400 Jupiter 
orbits.

The dynamical effects of secular forcing by the giant planet, along with  the effects of viscous stirring and gas drag on the planetesimals, are 
built into the initial conditions. This is done by first calculating the equilibirum eccentricity $e_{eq}$ due to viscous stirring and gas drag as a 
function of semimajor axis according to equation 12 of \citet{2002ApJ...581..666K}. Although {\sc ChaNGa} does not account for the effects of 
gas drag on the planetesimals, the viscous stirring timescale is long compared to our integration time and should not cause the velocity 
distribution to significantly evolve. The eccentricities of the bodies are drawn randomly from a Rayleigh distribution with a mode of $e_{eq}$, 
while the inclinations are drawn from a similar distribution with a mode of $e_{eq}$/2 \citep{1993MNRAS.263..875I}. The longitude of perihelia 
$\omega$ and the mean anomalies $M$ of the bodies are drawn from a uniform polar distribution.

To account for the effects of secular forcing, the eccentricity vectors of the planetesimals are first decomposed into real and imaginary 
components:

\begin{equation}\label{eq:hk}
	z = (h, ik) = e \, exp(i \omega)
\end{equation}

\noindent and a forced component is added to $h$ according to equation \ref{eq:eforced} (where we have set $\omega_{g}$ = 0).

\subsection{Time Stepping Scheme}\label{sec:timestep}

For the purposes of the integrator, there are two relevant timescales in this system. The first is the orbital dynamical time $\sqrt{a^3/
G M_{\odot}}$. All particles are placed on a fixed time step of $\Delta T$ = 0.01 yr. This is roughly 3\% of an orbital dynamical time at 
the inner edge of the planetesimal disk. From numerical tests with {\sc ChaNGa}, we have found that this time step size preserves 
the symplectic nature of the integration without being too computationally expensive. Although using $\Delta T$ = 0.01 yr keeps the 
integration symplectic, we find that a small amount of artificial precession is produced. We reduce the base time step size by an additional 
factor of 4 to the keep longitude of perihelia of planetesimals near the inner edge of the disk from drifting too far over the course of our 
integrations.

An additional timescale is set by the dynamical time of the planetesimals ($\sim 1/\sqrt{G \rho}$), which is about 45 minutes. To 
resolve the base time step and the dynamical timescale of planetesimals simultaneously, we use a two-tiered time stepping scheme. 
At the beginning of each time step, all bodies are placed on the orbital time step. A first pass of collision detection is then run in 
which the radii of all bodies are inflated by a factor of 2.5. Any bodies with imminent collisions predicted using the inflated radii are 
placed on a time step that is a factor of 16 smaller than the orbital time step. The purpose of two-tiered scheme is to properly resolve 
the gravitational interactions between any bodies that undergo a close encounter. This prevents the coarser base time step from 
reducing the effectiveness of gravitational focusing, while minimizing the additional computational expense.

\section{Results} \label{sec:results}

All five simulations are evolved for 5,000 years, which is roughly 400 orbits of the perturbing planet. Because the effects of the mean motion resonances are not built into the initial conditions, the simulations must be run long enough for the distribution of orbital elements to reach equilibrium near the resonances before the collision rate is measured. The libration timescale, which describes how long the critical angle (and also the semimajor axis and eccentricity) of a planetesimal in resonance takes to undergo a full oscillation is given by

\begin{equation}\label{eq:lib_time}
	T_{lib} = \frac{2 \pi}{3 q^{2} C_{r} n e^{\left| 2p + q \right|}},
\end{equation}

\noindent for small-amplitude librations. For all five sets of initial conditions used, $T_{lib} \sim$ 1,000 - 2,000 years for the 3:1 and 2:1 resonances. For this reason, we allow each simulation to run for 2,000 years before we begin tracking any collision statistics.

Add justification for the dust circularizing here as well.

\subsection{Varying the Eccentricity}

We begin by examining simulations $e_{lo}$, $n$ and $e_{hi}$. The positions of the planetesimals in the x-y plane after 5,000 years of integration is shown in figure \ref{fig:xy}. In all cases, the coordinate system is rotated so that the giant planet lies at $\theta = 0$ and the longitude of perihelion of the planet is shown by the dashed line. Resonances with the perturbing planet are visible via axisymmetric gaps. Upon close inspection of the last few simulation snapshots, these gaps appear to follow the planet in its orbit, rather than aligning themselves with the longitude of perihelion. A similar substructure reveals itself in \citet{2000Icar..143...45R} (see bottom panel of figure 3) and \citet{2016ApJ...818..159T} (figure 3). It is worth noting that \citet{2000Icar..143...45R} started with a completely cold planetesimal disk and a Jupiter mass planet on a circular, coplanar orbit. The presence of these features seems robust to the choice of initial conditions.

The effects of the resonances become much more apparent in semimajor axis-eccentricity space, which is shown in figure \ref{fig:ae}. In all cases, the 3:1, 2:1 and 5:3 resonances are readily visible as 'spikes' in the eccentricity that bend slightly inward (due to the conservation of $E_{J}$). In the $e_{hi}$ simulation, features also appear near the 5:2, 7:3 and 5:3 resonances. The absence of these features from the $e_{low}$ simulation can be explained by the fact that the strength of a resonance scales with $e^{q}$ \citep{1994PhyD...77..289M}.

Another important effect of the resonances is visible in figure \ref{fig:long_ph}, which shows the orientation of the longitude of perihelia of planetesimals in the disk. Inside of the resonances, orbits of planetesimals quickly precess and their orientations are effectively randomized. An important point to note is that this strong precession effect quickly disappears beyond the boundaries of the resonance. This turns out to be important to explain the nonaxisymmetric structure seen in figure \ref{fig:xy}, which will be addressed in more detail below.

Next, we examine the statistics of collisions resolved in each of the simulations. The 3D positions and velocities of the two colliding bodies are recorded to a table at the moment of impact. From these quantities, we also derive the Keplerian orbital elements of a collision from these quantities. First, we examine the semimajor axis of the first particle participating in each collision. During a collision, the 'first' particle is defined as the more massive of the two. However, nearly all of the collisions happen between the initial, equal-mass planetesimals. In this case, the distinction is set by the collision search algorithm and is rather arbitrary. In all of the plots where we show collision statistics, we have verified that using the 'first' or 'second' collider does not qualitatively change any of the features.

Figure \ref{fig:coll_hist_a} shows the semimajor axis distribution of the resolved collisions. We find that some of the features present in this and subsequent figures are highly sensitive to the number of bins and the location of the bin edges. For this reason, we construct a PDF of the collisions using a Kernel Density Estimate (KDE). We use the {\sc neighbors.KernelDensity} function from the {\sc sklearn} \citep{scikit-learn} package to construct our KDEs. For the kernel, we use a tophat function with a bandwidth of 0.02 AU.

Near some of the resonances, there are noticeable suppressions or enhancements of the local collision rate. This contrasts with the findings of \citet{2000Icar..143...45R}, who simulated a similar setup and found no discernible features near the MMRs. We attribute the differences to  a more conservative timestepping criterion in our simulations. The most prominent features appear as an enhancement to the collision rate near the 2:1 MMR and a decrease near the 3:1 MMR. At higher forced eccentricities, features near the 5:2, 7:3 and 5:3 resonances are also visible, due to the steeper sensitivity of higher order resonances to eccentricity \citep{1994PhyD...77..289M}.

We attribute the production of a bump vs a dip at certain resonances to changes in the gravitational focusing factor during two-body encounters. Closer to the star, the planetesimal disk starts off dynamically colder. Here, additional heating from the resonance acts to suppress the gravitational focusing cross section, which reduces the collision rate from the non-perturbed value. Further out in the disk, where the encounter velocity is already higher, the gravitational focusing cross section is already minimized. Any additional heating instead acts to simply increase the encounter rate between bodies, which enhances the collision rate. This explains the relative drop in the collision rate near the 3:1 MMR and the bumps that form near the 2:1 and 5:3 resonance. Although this could potentially serve as a useful diagnostic of the planetesimal size in a disk (because the mutual escape velocity sets how dynamically hot the bodies can get before gravitational focusing is suppressed), a direct measurement of the semimajor axes of the planetesimals is not possible. In addition, as will demonstrate next, collisions due to bodies in resonance end up having an insignificant effect on the final shape of the radial dust distribution.

Operating under the assumption that any dust produced in a collision will couple to the gas and circularize on a short timescale, we use the cylindrical distance at which a collision occurs to predict the radial structure of the dust profile. This is shown in figure \ref{fig:coll_hist_r}, which was constructed in a similar way to figure \ref{fig:coll_hist_a}. Most strikingly, the bump that was present near the 2:1 resonance is no longer present in cylindrical distance space, although it suddenly appears again in the $e_{hi}$ simulation. The dip that is present near the 3:1 MMR also disappears and instead presents as a bump in the $e_{hi}$ case. To determine how the resonant bodies actually contribute to the radial collision profile, we excluded collisions that fall between $2.495 < a < 2.505$ AU (near the 3:1 resonance) and between $3.2 < a < 3.35$ AU (near the 2:1 resonance), which is shown by the orange curve. Qualitatively, none of the bump or dip features present are changed by making this exclusion. This suggests that the features near resonance seen in semimajor axis space become 'smeared out' in cylindrical distance space due to the large spread in eccentricity and orbital orientation.

This also suggests that the features seen in figure \ref{fig:coll_hist_r} are being produced by bodies outside of resonance. To further understand this, we revisit the axisymmetric structure seen in figure \ref{fig:xy}. In figure \ref{fig:coll_polar_e}, we compare the radial collision profile to the polar structure seen in the $e_{low}$, $n$ and $e_{hi}$ simulations. In the lowest forced eccentricity case, the pileups near the edges of the resonances line up with the boundaries of the dip feature seen in the radial collision profile. This is especially prominent for the 2:1 resonance. As discussed previously, the pileups and gaps seen in polar coordinates follow the position of the giant planet in its orbit.

Although the axisymmetric pileup of bodies near the edges of resonances has been seen previously \citep{2000Icar..143...45R, 2016ApJ...818..159T}, we could not find a satisfactory explanation for why it happens and will provide one here. As mentioned previously, the circulation frequency of the critical angle slows as one approaches the edge of the resonance from the outside. Following the pendulum analogy, this is equivalent to reaching the point where the pendulum becomes suspended in the vertical position. When the critical angle stays relatively stationary during an encounter, the net torque will drive the longitude of pericenter towards the point of conjunction \citep{1976ARA&A..14..215P}.  This is the basic mechanism that causes isolated resonance to be stable. Inside of resonance, the critical angles librates about some value and this configuration is maintained. Just outside of resonance, however, the critical angle slowly drifts away, with the drift direction changing sign across the resonance. The net result of this is that bodies near the edge of the resonance experience a torque from that planet that temporarily aligns their pericenters with the orbital phase of the planet. After the encounter ends, differential rotation of the disk slowly destroys the configuration over part of an orbit. This explains why the concentration of planetesimals appears to 'follow' the the planet.

In a disk with no secular forcing, this dynamical phenomena would produce a 'dip' near the center of the resonance and a 'bump' at each edge in cylindrical distance space. When a forced eccentricity is introduced, the edges of the resonance, where the pileups are located, are able to explore a wider range of radii. If the the aphelion distance of the inner edge and the perihelion distance of the outer edge cross, we should expect to see the central dip feature in the collision profile start to disappear. This is exactly what appears to be happening in the middle panel of figure \ref{fig:coll_polar_e}. As the forced eccentricity is increased further, a region forms at the center of the resonance where planetesimals from both sides of the resonance spend time (although not simultaneously). When this occurs, a bump feature forms in the collision profile at this region. This is apparent in the bottom panel of figure \ref{fig:coll_polar_e}.

\begin{figure*}
\begin{center}
    \includegraphics[width=\textwidth]{figures/xy.png}
    \caption{Non axisymmetric gaps and rings appear near MMRs. More features appear at higher eccentricities. Gap features
    at $\theta$ = 0 and $\theta = \pi$ follow the giant planet in its orbit. (Update this to show long of peri of Jupiter)\label{fig:xy}}
\end{center}
\end{figure*}

\begin{figure}
\begin{center}
    \includegraphics[width=0.5\textwidth]{figures/ae.png}
    \caption{The MMRs produce spikes in the a-e plane. Between resonances, the nonzero eccentricity is caused by secular
    forcing by the planet. At larger eccentricities, the higher order resonances become more prominent because of the steeper
    scaling with $e$.\label{fig:ae}}
\end{center}
\end{figure}

\begin{figure*}
\begin{center}
    \includegraphics[width=\textwidth]{figures/long_ph.png}
    \caption{The MMRs also induce precession which overpowers the secular forcing. Inside of the resonances, orbits of
    planetesimals are effectively randomized.\label{fig:long_ph}}
\end{center}
\end{figure*}

\begin{figure}
\begin{center}
    \includegraphics[width=0.5\textwidth]{figures/coll_hist_a.png}
    \caption{In semimajor axis space, prominent features appear near the 3:1 and 2:1 MMRs. Near the 3:1, the collision
    rate decreases, while it gets enhanced near the 2:1. This is because gravitational focusing is already suppressed
    near the 2:1. Additional heating by the resonance increases the encounter rate.\label{fig:coll_hist_a}}
\end{center}
\end{figure}

\begin{figure}
\begin{center}
    \includegraphics[width=0.5\textwidth]{figures/coll_hist_r.png}
    \caption{The features near resonance that appear in cylindrical distance space appear qualitatively different.
    The orange curve shows collisions with bodies inside the 2:1 and 3:1 MMR excluded. This does not qualitatively
    change any of the plots, and we infer that the features seen here are produced by bodies outside of resonance.
    Most interestingly,  the highest eccentricity planet simulation produces bumps, rather than dips near the 2:1 and
    3:1 MMR.\label{fig:coll_hist_r}}
\end{center}
\end{figure}

\begin{figure*}
    \includegraphics[width=\textwidth]{figures/coll_polar_e.png}
    \caption{The features in the previous figure appear to match with the pileups of bodies near the edges of the
    resonances. The dashed and solid lines show the perihelion and aphelion distances for bodies at the inner
    and outer edges of the 2:1 and 3:1 resonances. The dip vs bump features seen in the previous figure appear
    to depend on whether the aphelia of bodies at the inner edge and perihelia of bodies at the outer edge of a
    resonance cross. When this does not occur, a cavity persists at the center of the resonance, which reduces the
    collision rate in that region. When the two distances overlap, bodies from both edges of the resonance spend some
    time in the middle, producing a bump in the collision rate.\label{fig:coll_polar_e}}
\end{figure*}

\begin{figure*}
    \includegraphics[width=\textwidth]{figures/coll_polar_m.png}
    \caption{Similar to the previous figure, except the eccentricity of the planet is kept constant and the mass is varied.
    This has the effect of changing the width of the resonances, without altering the relative apo or peri distances of
    bodies near the edges.\label{fig:coll_polar_m}}
\end{figure*}

\subsection{Collision Rates}\label{sec:coll_rates}

\subsection{Where Does the Dust End Up?}

\section{Influence of Jupiter's Mass and Eccentricity on the Radial Dust Profile}

\section{Observability of Dust} \label{sec:dust}

Aaron's CASA images go here

\section{Summary and Discussion} \label{sec:discuss}

\bibliography{references}

\clearpage

\end{document}
